\chapter{Fabrication of Interactive Objects} \label{ch:background}
  The construction of interactive objects using digital fabrication equipment
  has been a popular topic in the Human-Computer Interaction (HCI) literature in
  recent years. Researchers have proposed a variety of techniques for
  constructing objects that can sense, provide output to, and process user's
  interactions. Ballagas \etal present a comprehensive review of this design
  space~\cite{Ballagas:2018}, grouping previous endeavors by the type of
  mechanisms used to 3D-print interactive objects. This chapter, in contrast,
  aims to identify how this literature is progressing towards the \papf ideal.

  \section{\papf Literature Review}
    The process of interacting with a device, tangible or screen based, is
    comprised of three main steps. First, the device senses how it is being
    interacted with (e.g., mouse clicks, capacitive touch screen sensor).  Next,
    the device processes the provided interaction input. Last, the device
    provides an output to the user (e.g., visualizing on screen, change of
    physical shape). Mirroring this process, the structure of this section is
    divided into three main parts. First, I explore the literature relevant for
    constructing tangible devices that can sense user's interactions. Second, I
    review previous work that has tackled embedding processing capabilities to
    tangible devices. Last, I highlight endeavors that construct devices for
    output display.

    \subsection{Sensing User's Interactions} \label{sec:sensing}
      A variety of efforts have explored how to construct devices that can sense
      user's interactions using 3D-printers. Some early research on prototyping
      interactive objects focused on adding interactive functionality to the
      objects rather than on simple methods for fabrication. For example, some
      systems require assembling electronics and other components inside a
      printed shell \cite{Savage:2015ws, Savage:2013, Murray-Smith:2008,
      Hook:2014kp} and others require casting silicone \cite{He:2017, Rod:2017}.

      Other approaches require less assembly. Some research has detected changes
      in acoustical signals caused by user manipulation of geometry
      \cite{Savage:2015, Laput:2015, Li:2016}; however, the requirement for
      complex or movable geometry can mean considerable post-print effort for
      cleaning, assembling, and gluing.

      Some recent work has come much closer to the \pap ideal, enabling
      interactivity with significantly less or no post-print manipulation. One
      approach is to use multi-material printers to enable capacitive touch
      sensing \cite{Schmitz:2015, Schmitz:2019, Gotzelmann:2016} or optical
      sensing \cite{Willis:2012}; however, these approaches require attachment
      of multiple points of circuitry or optic sensors to operate, and the size
      of object is limited. Another optical approach is to use computer vision
      to detect user interaction \cite{Shi:2016a}; however, cameras are prone to
      problems with occlusion (i.e., touches on the back of an object cannot be
      detected, nor can touches hidden by the hand itself), and it is difficult
      to differentiate touching from merely being close to the object.

      Several projects require nearly no post-print manipulation.
      Touch \&~Activate \cite{Ono:2013} used an affixed microphone and speaker
      to detect how acoustic sweeps were changed by user touch; this technique
      worked with many objects, including off-the-shelf ones, but required a new
      machine-learning model to be trained for every object.  Tickers and Talker
      \cite{Shi:2016} used centimeter-scale physical markers which made unique
      sounds when plucked, but significantly impact the geometry of the object.
      INTACT \cite{Hudin:2016} uses a 3D~model of an object placed on a 6-axis
      force sensor to mathematically determine where the user is touching. While
      it can sense touch with high precision, objects are limited in size to
      around 20\,cm, and require recalibration after moving.

    \subsection{Computing} \label{sec:computing}
      Previous approaches for digitally fabricating interactive objects
      have often used existing computing devices (e.g., computers,
      phones) connected to the fabricated objects in order to drive the
      interactions. Researchers have explored the use of
      acoustic~\cite{Savage:2015, Tejada:2018, Shi:2016},
      pneumatic~\cite{Tejada:2020, Vazquez:2015, Ou:2016},
      electronic~\cite{Schmitz:2019, Savage:2014,Schmitz:2015}, and
      optic~\cite{Willis:2012, Savage:2013} techniques to fabricate
      objects that can respond to user's interactions. While the
      computation required to drive most of these approaches could
      theoretically be implemented using smaller computing devices and
      embedded inside the final 3D-printed object, this would require
      significant engineering expertise from the designer to assemble
      circuits or printed parts.

      Other approaches ``store'' the results of the interactions for
      later processing. For example, in Off-line Sensing, Schmitz \etal
      introduce an approach to develop one-time 3D-printed sensors using
      liquids to memorize the results of interactions. These sensors can
      then be read using a smartphone~\cite{Schmitz:2018}. Similarly,
      Iyer \etal developed 3D-printable structures capable of storing
      linear, and rotational interactions using a coil-like structure.
      These devices can transmit the stored interactions wirelessly once
      in range~\cite{Iyer:2018}. While successful, these efforts require
      manual intervention during deployment. They require designers to
      assemble multiple 3D-printed parts~\cite{Iyer:2018}, or carefully
      pour liquid into the constructed objects~\cite{Schmitz:2018}.

      Last, are the endeavors that embed computation inside the constructed
      object. While this has been traditionally achieved by assembling custom
      electronics and other components inside a 3D-printed
      shell~\cite{Murray-Smith:2008}, other efforts aim to simplify this process
      by leveraging existing computing devices and embed them inside the
      fabricated object. Acoustruments, for example, makes use of a smartphone
      embedded inside the object to enable rich input
      modalities~\cite{Laput:2015}. Pineal builds on this concept and makes use
      of smartwatches in addition to smartphones in order to augment objects of
      various dimensions while also providing rich output to the
      user~\cite{Ledo:2017}. Other endeavors make use of mechanical computing to
      augment digitally fabricated objects with simple logic processing. Ion
      \etal developed logic cells that can be embedded inside 3D-printed
      objects, enabling them to compute simple logic operations~\cite{Ion:2017},
      while Song \etal devised a technique for manufacturing micro-mechanical
      logic gates using digital fabrication equipment~\cite{Song:2019}. Despite
      their success, these approaches share a common limitation: they rely on
      complex or lengthy assembly processes in order to be realized.

    \subsection{Providing Output} \label{sec:output}
      A large body of work in the HCI and material science communities have
      explored techniques for creating objects, and materials that can provide
      output of computations either acoustic, visual, or haptic. This segment of
      devices stray the most from being \pap. Some efforts require complex
      circuitry to be enabled~\cite{Slyper:2012, Groeger:2016}, assembly of
      parts~\cite{Ion:2017}, prohivitive fabrication pipelines~\cite{Kong:2014,
      Peng:2016, Brockmeyer:2013}, or a combination of these~\cite{Vazquez:2015,
      MacCurdy:2016, Neidlinger:2017}. An effort that approaches to being \pap
      is Acoustic Voxels~\cite{Li:2016}. In it, the authors propose a set of
      acoustic filters that modify incoming sounds depending on their
      configuration. The objects augmented using these technique can be printed
      as a single structure, and do not require electronics or calibration of
      machine learning models.

  \section{Discussions \& Observations}
    Despite the strides made by previous explorations on streamlining the
    construction of tangible devices, the literature nonetheless strays from the
    \papf ideal. This section discusses the advances made by previous work
    through the lens of \papf.

    Section \ref{sec:sensing} shows how previous work has progressed from
    requiring complex assemblies and fabrication pipelines to enable sensing
    user's interactions on fabricated objects, to almost seamless approaches
    that require very little intervention post-fabrication. However, despite
    these progresses, there most streamlined processes for constructing objects
    that can sense user's interactions still present significant friction. For
    example, efforts like Touch~\&~Activate~\cite{Ono:2013}, and Tickers and
    Talkers~\cite{Shi:2016}, while not requiring any complex assemblies or
    prohivitive fabrication pipelines, they require per-object calibration of
    machine learning models--a task arguably more complex for non-expert users. 

    Contrary to techniques that enable sensing of interactions on tangible
    devices, efforts that enable computation in 3D-printed objects diverge
    significantly from the \papf ideal. Efforts that aim to mechanically embed
    computation into tangible devices, such as \cite{Ion:2017}, require lengthy
    assemblies of numerous printed parts, while endeavors that ``store'' the
    results of interactions require the careful pouring of liquids into
    constructed objects~\cite{Schmitz:2018}, both tasks demanding significant
    domain knowledge from the designer. The technique that approaches closer to
    the \papf ideal is Pineal~\cite{Ledo:2017}. While requiring manual assembly
    post-print, this technique provides designers guidelines to designers how to
    assemble the printed parts and mobile devices---an interesting way of
    decreasing the complexity for constructing tangible devices.
    
    Last are the techniques that construct tangible devices capable of
    displaying the output of computations to their users. When compared to
    sensing approaches, techniques that construct tangible devices for
    displaying the output of computations to their users remains unexplored. In
    their review~\cite{Ballagas:2018}, Ballagas \etal identified only 18 such
    efforts, compared to the 79 that enable the construction of tangible devices
    to sense user's interactions. The techniques explored, however, stray the
    farthest from the \papf ideal, requiring lengthy assembly of
    parts~\cite{Ion:2017, Willis:2012}, and electronics~\cite{Murray-Smith:2008}.

    In closing, the literature exploring the construction of tangible devices
    using 3D printers trends towards the \papf deal, forgoing elaborate
    assemblies for a more accessible construction process. This accentuates the
    importance of \papf. Despite the literature trending towards more
    \pap-friendly techniques, the literature does does not yet provide a
    technique that fulfills the requirements for \papf, previously discussed.
    This thesis introduces four distinct approaches for constructing immediately
    usable tangible devices. These techniques aim to inch closer to the ideal
    future of on-demand fabrication of tangible devices.