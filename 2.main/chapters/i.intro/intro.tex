\chapter{Introduction}
  \section{The Inequalities of Interactivity}
    The ubiquity of personal computing brought with it an increasing
    accessibility in creating interactive digital experiences.  In \red{XX}
    years, we have gone from programming computers using punched cards, to
    creating immersive virtual environments by dragging virtual elements and
    dropping them in their respective
    places~\footnote{\href{https://unity.com}{Unity}}.  Currently, you can
    create an online store without knowing anything about web
    programming~\footnote{\href{https://www.squarespace.com}{Squarespace}},
    creating body-based augmented reality applications using visual
    programming.\cite{Pohl:2020}, and \red{something else.}

    Contrary to this ease of crafting interactive digital experiences,
    constructing interactive physical objects remains a challenging
    undertaking. While the construction, and usage of interactive digital
    interactive experiences has been researched for over forty
    years~\cite{CHI, UIST}, no long-term research has explored ways to
    construct interactive experiences in the physical world. Because of this,
    constructing physical interactive experiences still require significant
    domain knowledge, and remains out of reach for other-domain experts.
   
    \red{Closing paragraph.}

  \section{The Personal Fabrication Revolution} \label{sec:fab-revolution}
    A promising remedy that could tip the making scales to balance is
    personal fabrication. While the technologies that enable personal
    fabrication \red{machines} are almost fifty years old, these advances are
    now accessible and reliable enough to be used by consumers and hobbyists.

    In their respective books Gershenfeld and Anderson describe a future where
    we all have a fabrication machine in our homes~\cite{Gershenfeld:2005,
    Anderson:2012}.  Homologous to our current use of 2D paper printers,

  \section{Broken Promises of the Fabrication Revolution}
    While fabrication machines have become accessible, and reliable enough to
    make the jump from industry and research laboratories to the hands of
    consumers and hobbyists, we are still far away from the fabrication
    revolution Gershenfeld and Anderson professed. While most recent market
    analysis place 3D printer market penetration at almost 35\%, 77\% are
    relegated to the industry~\cite{}. This same analysis reveals that the main
    use for 3D printers is prototyping future products, rather than creation of
    usable things. The results of this reports uncover two main

    I believe that the lackluster adoption of personal fabrication machines is
    due to two main factors. First, current fabrication equipment is limited to
    what types of objects it can produce. Current fabrication devices can
    construct a variety of shapes using different types of plastics and metals,
    which, while interesting and useful, is far from the replicators shown in
    science fiction (Figure \ref{fig:replicator}), or the machines capable of
    building other machines presented by Gershenfeld~\cite{Gershenfeld:2005}.

    \begin{figure}[h]
      \centering
      \includegraphics{example-image-duck}
      \label{fig:replicator}
      \caption{Tea. Earl Grey. Hot}
    \end{figure}

    The second issue relates to the expertise required to construct
    \red{interesting} objects. While current personal fabrication equipment is
    able to produce a variety of \red{intersting} objects from prosthetic arms
    for children (Figure \ref{fig:expertise-prosthetic}) to intricate jewelry
    pieces (Figure \ref{fig:expertise-jewelry}), the construction of these
    objects require ample domain knowledge that consumers might not possess.
    This leaves very limited options for low-entry uses for fabrication devices.

    \begin{figure}[h]
      \centering
      \subfloat[3D-printed Jewelry]{
        \label{fig:expertise-jewelry}
        \includegraphics[width=0.43\textwidth]{example-image-duck}
      }
      \subfloat[3D-printed Prohstetics]{
        \label{fig:expertise-prosthetic}
        \includegraphics[width=0.43\textwidth]{example-image-duck}
      } \hfill
      \subfloat[3D-printed Speakers]{
        \label{fig:fig2}
        \includegraphics[width=0.43\textwidth]{example-image-duck}
      }
      \subfloat[3D-printed Actuators]{
        \label{fig:fig2}
        \includegraphics[width=0.43\textwidth]{example-image-duck}
      }
      \label{}

      \caption{Interesting stuff require expertise.}
    \end{figure}

  \section{Towards On-Demand Fabrication of Interactive Devices}
    Inspired by the personal fabrication \red{dream} described in Section
    \ref{sec:fab-revolution}, this thesis explores a potential future where
    not only objects are fabricated on-demand, but interactive devices.

    \red{Something else should go here. Maybe on what we gain by doing this,
    why is this worth doing and important?  I should cite Everyday Makers
    here.  Non-experts wanna print interactive stuff, not just shapes}

    We tackle two of the main problems with fulfilling the fabrication
    revolution: we help non-experts to make useful stuff, and they require no
    previous knowledge to do so.

    \subsection{Challenges}
      Recent Human-Computer Interaction literature has explored numerous
      techniques for constructing interactive devices using digital fabrication
      equipment~\cite{Ballagas:2018}. Despite the various interaction modalities
      these efforts enable, their adoption has been limited to research
      laboratories. I postulate that this is due to two main factors:

        \subsubsection*{Convoluted Post-print Activities}
          Some of the proposed \red{mechanisms} to print interactive devices
          require intricate operations to be carried out after the device has
          finished printing. These can take the form of removal of support
          material~\cite{Laput:2015}, connecting multiple circuit
          ports~\cite{Schmitz:2019}, assembly of physical
          parts~\cite{Savage:2015}, circuits~\cite{Murray-Smith:2008}, or
          calibration of machine learning algorithms~\cite{Ono:2013}.

          These processes require significant engineering expertise.

        \subsubsection*{Prohibitive Fabrication Pipelines}
          Other proposed \red{mechanisms} require \red{prohibitive} fabrication
          processes to be enabled. Some require specialized printers~\cite{}, or
          silicone casting~\cite{He:2017}

          These processes require techniques, or equipment that are not in reach
          for consumers or hobbyists.

        \subsubsection*{Modifications to the object external geometry}
      
    \subsection{\papf}
      To address the aforementioned challenges, this document proposes \papf:
      a digital fabrication paradigm where interactive devices are
      immediately usable after construction.

      \red{How do I do \pap?}
        Air powered techniques. Why?

      \subsubsection*{Contributions of this Thesis}
        This concept directly addresses the two main challenges from adoption of
        personal fabrication equipment discussed above. First, it provides
        non-experts with a variety of options to fabricate using off-the-shelf
        personal fabrication equipment. Second, it does so by requiring no domain
        expertise to construct these interactive devices.

        \red{Something here how I do input, output, and logic. Maybe mix it with
        intros to the papers}