\chapter{Introduction}
  \section{The Inequalities of Interactivity}
    We are surrounded by products with dedicated physical interfaces, like
    steering wheels, musical instruments, and game controllers. While the advent
    of screen-based \red{devices} has led to a rise in touch, and mouse based
    applications, there are some benefits of physicality that simply do not
    translate to screen-based \red{experiences}~\cite{klemmer:2006}, like
    hands-free interaction, speed, virtuosity, and control. This is the reason
    why we see most experts use dedicated physical interfaces instead of their
    screen-based counterparts (e.g., airplane pilots, musicians, gamers).

    Because the increased presence of screen-based devices, significant advances
    has been made to streamline the creation of screen-based \red{interactive
    experiences}.  In 60 years, the state-of-the-art has gone from programming
    computers using punched cards, to creating immersive virtual environments by
    dragging virtual elements and dropping them in their respective
    places~\footnote{\href{https://unity.com}{Unity}}.  Currently, you can
    create an online store without knowing anything about web
    programming~\footnote{\href{https://www.squarespace.com}{Squarespace}}, or
    creating body-based augmented reality applications using visual
    programming~\cite{Pohl:2020}.

    Contrary to this ease of crafting interactive digital experiences,
    constructing interactive physical objects remains a challenging
    undertaking. While the construction, and usage of interactive digital
    interactive experiences has been researched for over forty
    years~\cite{CHI:, UIST:}, no long-term research has explored ways to
    construct interactive experiences in the physical world. Because of this,
    constructing physical interactive experiences still require significant
    domain knowledge, and remains out of reach for other-domain experts. Although
    existing research has explored electronic toolkits to lower the threshold of
    making physical, interactive devices~\cite{Greenberg:2001, Arduino:}, these
    require the assembly of electronic sensors, actuators and physical parts to
    construct an interactive device.

    While the process of assembling circuits and printed parts may seem
    straightforward for some, this is an intricate process because, and remains
    out of the reach of other domain experts (e.g., doctors, architects). This
    stands in contrast to the easiness of creating interactive applications for
    screen-based devices, where little to no domain expertise is needed.
   
  % Play this up more
  \section{The Personal Fabrication Revolution} \label{sec:fab-revolution}
    A promising remedy that could tip the making scales to balance is
    personal fabrication. While the technologies that enable personal
    fabrication \red{machines} are almost fifty years old, these advances are
    now accessible and reliable enough to be used by consumers and hobbyists.

    In their respective books Gershenfeld and Anderson describe a future where
    we all have a fabrication machine in our homes~\cite{Gershenfeld:2005,
    Anderson:2012}.  Homologous to our current use of 2D paper printers, the
    authors describe how in this potential future we would print our objects,
    rather than buy them, dawning an age of on-demand fabrication.

  \section{Broken Promises of the Fabrication Revolution} \label{sec:broken-promises}
    While fabrication machines have become accessible, and reliable enough to
    make the jump from industry and research laboratories to the hands of
    consumers and hobbyists, we are still far away from the fabrication
    revolution Gershenfeld and Anderson professed. While most recent market
    analysis place 3D printer market penetration at almost 35\%, 77\% are
    relegated to the industry~\cite{}. This same analysis reveals that the main
    use for 3D printers is prototyping future products, rather than creation of
    usable things. 
    % The results of this report uncover two main findings:

    I believe that the lackluster adoption of personal fabrication machines is
    due to two main factors. First, current fabrication equipment is limited to
    what types of objects it can produce. Current fabrication devices can
    construct a variety of shapes using different types of plastics and metals,
    which, while interesting and useful, is far from the replicators shown in
    science fiction (Figure \ref{fig:replicator}), or the machines capable of
    building other machines presented by Gershenfeld~\cite{Gershenfeld:2005}.

    \begin{figure}[h]
      \centering
      \includegraphics{example-image-duck}
      \label{fig:replicator}
      \caption{Tea. Earl Grey. Hot}
    \end{figure}

    The second issue relates to the expertise required to construct usable
    objects. While current personal fabrication equipment is able to produce a
    variety of interesting objects from prosthetic arms for children (Figure
    \ref{fig:expertise-prosthetic}) to intricate jewelry pieces (Figure
    \ref{fig:expertise-jewelry}), the construction of these objects require
    ample domain knowledge that consumers might not possess.  This leaves very
    limited options for low-entry uses for fabrication devices.

    \begin{figure}[h]
      \centering
      \subfloat[3D-printed Jewelry]{
        \label{fig:expertise-jewelry}
        \includegraphics[width=0.43\textwidth]{example-image-duck}
      }
      \subfloat[3D-printed Prohstetics]{
        \label{fig:expertise-prosthetic}
        \includegraphics[width=0.43\textwidth]{example-image-duck}
      } \hfill
      \subfloat[3D-printed Speakers]{
        \label{fig:fig2}
        \includegraphics[width=0.43\textwidth]{example-image-duck}
      }
      \subfloat[3D-printed Actuators]{
        \label{fig:fig2}
        \includegraphics[width=0.43\textwidth]{example-image-duck}
      }
      \label{}

      \caption{Interesting stuff require expertise.}
    \end{figure}

  \section{Towards On-Demand Fabrication of Interactive Devices} \label{sec:on-demand}
    Despite the utility of custom-made three-dimensional shapes, recent work has
    highlighted that ``everyday makers'' are most interested in fabricating
    objects they can interact with~\cite{Shewbridge:2014}. Inspired by these
    findings, and the personal fabrication ideal described in Section
    \ref{sec:fab-revolution}, this thesis explores a potential future where
    digital fabrication equipment not only enables the construction of custom
    three-dimensional shapes, but of interactive devices as well.

    The creation of these interactive devices should be as natural and automatic
    as possible. An ideal \red{form} would be homologous to today's paper
    document creation pipeline: users create a document using a piece of
    software (e.g., word processor), and proceed to print it using a 2D desktop
    printer. Once the paper has finished printing, this process is completed;
    there is nothing else to do.

    \subsection{Challenges}
      The concept of constructing interactive devices using digital fabrication
      equipment is not new. A multitude of research endeavors in the
      Human-Computer Interaction (HCI) community have explored techniques to
      fabricate interactive devices using 3D-printers or other digital
      fabrication equipment~\cite{Ballagas:2018}. However, despite these
      advances, the introduced techniques have yet to escape the research
      laboratories and move on into the masses. This is due to two main factors:

      \subsubsection*{Convoluted Fabrication Pipelines}
        A paramount research challenge researchers need to resolve is the
        democratization of \red{techniques}. Revisiting the paper document
        metaphor, when the same document is printed in different types of 2D
        printers, the outcome is the same. So it should be when constructing
        interactive devices: the method of construction should not affect the
        capability of the device. Additionally, we should strive to automate the
        construction process as much as possible, opting to use accessible
        fabrication machines and devices over manual labor. We should also avoid
        using non-accessible materials.

      \subsubsection*{Complex Post-Print Activities}
        An additional challenge is the removal of complex post-print activities.
        Current fabrication pipelines for the construction of interactive
        devices rely on convoluted post-print activities in order to fabricate
        these devices. These post-print activities can take the shape of
        assembly of circuits, printed parts, complex removal of support
        material, or calibration of per-object, or per-user machine learning
        models.
      
    \subsection{\papf}
      As a way towards the potential future of on-demand fabrication of
      interactive devices described in Section \ref{sec:on-demand}, this thesis
      introduces \papf: a digital fabrication paradigm where interactive objects
      are printed instead of assembled. \pap fabrication techniques construct
      objects that are immediately usable after fabrication, meaning there is no
      assembly of parts, circuits, or calibration of machine learning models
      needed. Additionally, techniques that are \pap-friendly, are constructable
      using off-the-shelf fabrication equipment and materials.

      The concept of \papf directly addresses the shortcomings of the personal
      fabrication revolution described in Section \ref{sec:broken-promises}:
      \red{lack of things to fabricate using plastic}, and the necessity of
      domain knowledge to fabricate interesting things. The techniques
      presented in this thesis do not require special materials, or printers;
      increasing the vocabulary of what can be constructed using simple
      plastics. Additionally, because these techniques also do not require any
      post-print activities or specific domain knowledge to be enabled, and all
      provide an intuitive design environment for the creation of interactive
      devices, this increases the target audience that can make use of these
      techniques.

      \newpage
      \subsubsection*{Contributions of this Thesis}
        This thesis focuses on the easy construction of interactive devices
        without requiring significant post-fabrication activities (e.g.,
        assembly of parts, circuits, calibration of machine learning models), or
        specialized fabrication pipelines.  With my work, I enable the creation
        of devices in all the sections of the interaction stack (Figure
        \ref{fig:stack}): devices that can sense, process, and provide output to
        user's interactions.

        \begin{figure}[h]
          \centering
          \includegraphics{example-image-duck}
          \label{fig:stack}
          \caption{Input - Processing - Output}
        \end{figure}
        
        An overarching theme of this work is \emph{air}. All of the work
        presented in this thesis is powered by air, being from a constant,
        pressurized air source (Chapters \ref{ch:airtouch}, \ref{ch:morphees},
        and \ref{ch:airlogic}), or from the user's lungs (Chapter
        \ref{ch:blowhole}). The use of air as a \red{medium} to enable the
        construction of interactive devices was motivated by two factors. First,
        air and fluid behaviors have been extensively explored in physics, which
        gives us a ample understanding how air behaves in a variety of
        situations, and which ones of these lend themselves to digital
        fabrication pipelines.  Allowing us then to use these concepts to
        construct pre-trained machine learning models, or use their respective
        mathematical equations. Second, I can then embed internal structures
        inside three-dimensional models that leverage these behaviors without
        greatly modifying the object's external geometry. A summary of the
        individual contributions of each effort is presented below:

        \textsc{\at} extends the possibilities of pneumatic sensing on
        fabricated objects by enabling sensing users interactions not only on
        soft objects, but in rigid ones as well. \emph{\at} is based on basic
        principles of fluid dynamics that relate pressure to discharge area:
        when there is a change in the discharge area of a system, the pressure
        will vary in return. Objects augmented with \emph{\at} can sense touch
        interactions in up to 12 individual locations, and are printed as a
        single structure. These objects are fabricated as a single structure
        without requiring any assembly of parts or circuits. Additionally, they
        make use of pre-trained machine learning models, so no per-object or
        per-user calibration is necessary.
        
        \textsc{\bh} continues to explore techniques for enabling sensing of
        user's interactions on 3D-printed objects by addressing one of
        \emph{\at's} principals shortcomings: the need for a constant air source
        and barometric pressure sensors. Based on the principle of acoustic
        resonance, \emph{\bh} embeds resonant cavities in the interior of the
        model that, when gently blown on, emit an identifiable sound.
        Interactive devices constructed using \emph{\bh} are fabricated as a
        single structure using commercially-available printers and materials,
        and make use of a mathematical model to identify blow locations based on
        the frequency of the generated sound.

        Switching gears to computing, \textsc{AirLogic} aims to answer the
        question ``how can we 3D-print devices that can think?'' Inspired by
        research from the Cold War, we developed a series of widgets that can
        not only sense user's interactions, and provide output to them, but also
        compute simple logic operations (e.g., \texttt{AND, OR, NOT}, and
        \texttt{XOR}).

        Last, with \textsc{MorpheesPlug} I explore the construction of devices
        that can provide physical output. This work was inspired by recent
        research on shape-changing interfaces, where physical interfaces change
        their shape depending on their intended use. We approached this concept
        from a fabrication standpoint, creating a set of pneumatically-actuated
        widgets that can represent a variety of changes in shape. These widgets
        are fabricated as a single structure, using off-the-shelf printers and
        materials.