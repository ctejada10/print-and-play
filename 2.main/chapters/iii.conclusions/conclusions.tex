\chapter{Conclusions} \label{ch:conclusion}
  In closing, in this thesis I have introduced the concept of \papf: a
  fabrication paradigm where tangible, interactive devices are fabricated
  instead of assembled. To explore different ways we can enable \papf, I have
  introduced four techniques: \at, \bh, \al, \mp; each of these tailored to
  tackle a specific facet of tangible devices. With \at and \bh, I explore novel
  ways to provide input to tangible devices. Continuing, \al investigates
  interesting ways to represent encapsulate logic in \red{tangible devices}.
  Last, \mp researches novel ways of constructing devices that can change in
  shape, and provide physical output to input and computations.