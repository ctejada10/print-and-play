\chapter{Conclusions} \label{ch:conclusion}
  In closing, in this thesis I have introduced the concept of \papf: a
  fabrication paradigm where tangible, interactive devices are fabricated
  instead of assembled. To explore different ways we can enable \papf, I have
  introduced four techniques: \at, \bh, \al, \mp; each of these tailored to
  tackle a specific facet of tangible devices. With \at and \bh, I explore novel
  ways to provide input to tangible devices. Continuing, \al investigates
  interesting ways to represent encapsulate logic in tangible devices, without
  requiring custom electronics.  Last, \mp researches novel ways of constructing
  devices that can change in shape, and provide physical output to input and
  computations.

  The work of this thesis illustrates that a promising approach for enabling
  \papf is to embed custom interior structures in three-dimensional models
  to leverage fluid, in this case air, behavior to enable sensing, logic
  processing, and output display. I have demonstrated that using this
  concept we can construct tangible, interactive devices that can sense,
  process, and display output using consumer-grade 3D printers that are
  immediately usable after fabrication.

  The suite of tools I present in this thesis, the techniques used to enable
  them, and the space they explore, will hopefully empower designers to
  construct tangible, interactive devices using 3D printers, and inspire
  researchers to continue exploring different ways to enable \papf.