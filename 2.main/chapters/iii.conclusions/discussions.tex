\chapter{Discussions \& Implications} \label{ch:discussion}
  This thesis has presented a set of techniques that enable the construction of
  interactive, tangible devices without requiring any domain experience (e.g.,
  assembly of printed parts and circuits, or calibration of machine learning
  models). To conclude, I discuss how each of the proposed techniques moves
  the state-of-the-art closer to the \papf ideal, and highlight some paths for
  future work.

  \section{Discussion of Projects}
    The main takeaway from this thesis is this: constructing tangible,
    interactive devices is hard---significantly harder than their on-screen
    counterparts---because to do so it is necessary to be well-versed in
    software, hardware, and digital fabrication. Ideally, digital fabrication
    equipment should abstract most, if not all, of these complications and
    enable designers to fabricate tangible, interactive devices that are
    immediately usable after fabrication. This ideal scenario is what I call
    \emph{\papf}: a fabrication paradigm where tangible devices are printed, not
    assembled.

    In this thesis, I have presented four \pap techniques: \al, \bh, \al,
    \mp; each aimed to address specific barriers in sensing, processing, and
    providing output to user's interactions. While this thesis has been aimed
    for the understanding of the general scientific community, the design and
    schematics of each of these projects have been open-sourced upon the
    publication of their respective manuscripts, enabling not only researchers
    but non-expert designers to experiment with the proposed techniques.

    I envision that, mirroring the process of creating graphical user
    interfaces, a \papf toolkit will abstract the interactive mechanisms from
    their respective techniques. For example, designers can specify ``a touch
    location here'', ``a linear actuator there'', and the system would create
    the desired tangible device, with the specified interaction modes.
    Continuing, once the tangible device has been designed successfully, the
    toolkit must be able to aid with the required software deployment. For
    tangible devices that can sense user's interactions, I envision this to be
    approached in two manners. For non-expert users, the \papf toolkit can
    allow some form of programming by example, where designers specify locations
    in the model and link them to specific actions; similarly to how \bh's
    design environment operates. More expert users, on the other hand, can
    obtain pre-trained machine learning models to embed into their applications.

    In addition to contributing to the digital fabrication community, the work I
    present in this thesis builds on the larger trend in computer science,
    specifically Human-Computer Interaction, where computing devices are imbued
    with domain expertise to support users. Although previous explorations of
    this concept in a digital fabrication setting have been aimed at preserving
    user creativity and agency~\cite{Zoran:2013}, the work discussed above aims
    to abstract the complexity of creating tangible devices into the fabrication
    process using 3D printers. By removing the complexities of constructing
    tangible, interactive devices and instead abstracting them into 
    fabrication pipelines, we can enable non-expert designers and hobbyists to
    create on-demand tangible devices without requiring extensive domain
    expertise, making the fabrication revolution ideal closer to reality.

    A potential development that might change the performance of the projects
    discussed above is the advent of new fabrication machines, or the increased
    accessibility of higher-end, current ones. While some of the barriers that
    previous endeavors face may be overcome by improvements in digital
    fabrication equipment (e.g., complex support material
    removal~\cite{Laput:2015}, or manual fabrication procedures~\cite{He:2017}),
    others efforts present intrinsic barriers in the way they enable the
    construction of tangible devices. For example, in Sauron~\cite{Savage:2013},
    Savage \etal use video cameras placed in the interior of the object to track
    user's interactions.  Although fabrication methods will improve over time,
    to create Sauron-enabled objects, designers will be required to embed
    cameras, mirrors, and reflectors inside their fabricated objects, with all
    the complications this process entails. Another effort that presents similar
    barriers is \texttt{./trilaterate}~\cite{Schmitz:2019}, by Schmitz \etal
    While their fabricated objects are printed as a single structure using
    multi-material printers, the chosen way to identify user's interactions,
    trilateration, will require post-print calibration by user, and by object.
    This stands in contrast to the techniques included in this thesis, where the
    advent of new fabrication technologies will only improve their performance.
    For example, higher resolution printing can enable \at to embed more
    touch-sensitive locations throughout the model, breakthroughs in flexible
    materials can allow \mp to construct a wider range of shapes without the use
    of internal support, and the use of acoustic metamaterials like the ones
    presented in~\cite{Haberman:2016} could allow \bh to render a wider range of
    frequencies for its interactive locations.

    Last, in this thesis I have presented a series of projects that enable \papf
    using air-powered objects, via custom internal structures. As mentioned
    previously, the use of air as a driving mechanism for the proposed
    techniques was motivated by two main factors. First, the ample study on
    fluid behavior, allows for the design and construction of custom interior
    structures to leverage various physical phenomena for constructing tangible
    devices that can sense, process, and provide output to user's interactions.
    Second, while these principles of fluid behavior are applicable to most
    fluids (e.g., water, oil, other gases), the use of air permits for a
    ``cleaner'' operation. Compressed air sources are widely available, and used
    air can be safely discharged into the atmosphere. This does not mean,
    however, that only air-powered approaches can be \pap.  As mentioned
    previously, any technique that can allow the construction of tangible,
    interactive devices without the need of complex post-print activities, or
    prohibitive fabrication pipelines is a \papf technique. For example, recent
    work from Schmitz and colleagues present a very interesting way to tackle
    assembly: using magnets and conductive filament~\cite{Schmitz:2021}. The
    objects fabricated using this technique are constructed in two parts: an
    ``Oh Snap!  board'' which houses a microcontroler with the logic, and the
    printed object itself.

  \section{Directions for Future Work}
    Throughout this thesis, I have laid out possible directions future
    researchers can improve each project. Moving from per-project improvements,
    this section aims to discuss how \papf can be improved by future
    explorations.

    \subsection{Support for multi-material printing}
      All through the projects that make up this thesis, I have made a conscious
      decision to construct tangible, interactive devices using single-material
      fabrication pipelines in order to increase the potential audience of each
      technique. Future efforts researching new \pap techniques for constructing
      tangible devices can explore the use of multi-material fabrication
      pipelines. Significant work has already been carried out by
      Schmitz~\cite{Schmitz:2019a} using a combination of conductive and
      nonconductive materials to construct tangible devices. While these efforts
      are not truly \pap, as they require post-print activities including
      carefully pouring liquids into the object, machine learning
      calibrations, or attaching multiple points of contact, they highlight the
      promising possibilities of using multi-material fabrication pipelines to
      construct interactive objects.

      In addition to using conductive and nonconductive material to construct
      tangible devices, future work can explore the use of metamaterials to
      create new composite materials, tailored for specific functions.  The use
      of multi-material fabrication pipelines has the potential for enriching
      the vocabulary of interaction modalities, while still remaining \pap.  For
      example, the use of auxetic metamaterials can enable a variety of haptic
      feedback when interacting with the printed device.

    \subsection{Support for other interaction modalities}
      The work discussed above focuses on demonstrating the possibilities of
      constructing tangible devices with little designer intervention
      post-print, rather than exploring a variety of interaction modalities. An
      interesting direction for future work is to investigate how the \papf
      principles can be materialized with various interaction modalities like
      deformation, and environmental interactions, as well as expand the
      interaction modalities already explored in this thesis.

      All the sensing techniques presented in this thesis are tailored for
      interaction on hard tangible devices. An interesting avenue for future
      work to explore is the implementation of \pap techniques that enable
      reliable deformation sensing in soft, and flexible objects. Regardless of
      the technique chosen to implement this interaction modality (e.g.,
      pneumatic, acoustic, or electric sensing), I believe the main challenge in
      implementing this interaction modality in a \pap way will be the removal
      of per-object calibration of machine learning models. Because of each
      object's varying geometry, interacting with different object should yield
      different results. Inspired by \at's success in reusing machine learning
      models for tangible devices of varying geometries, future work can explore
      the use of custom, stable interior structures for deformation (e.g.,
      press, squeeze) sensing.

      Continuing, I have presented two techniques for the construction of
      efforts that are able to respond to touch interactions: \at and \al. While
      successful, these efforts can only reliably sense single touch locations.
      I have laid out guidelines on how to pneumatically sense more than one
      location in Chapter \ref{ch:airtouch}, but this approach limits the number
      of interaction locations if a multi-touch setup is desired. Future
      explorations can investigate other \pap-friendly techniques for
      constructing tangible devices capable of sensing touch interaction in
      various locations.

      Conversely, with \al I explored ways to physicalize the output of logical
      operations. These outputs can also be expanded by future work. For
      example, \al presents a vibrotactile motor to provide haptic feedback to
      its users; future work can explore other haptic interaction modalities as
      output, like temperature. Similarly, \al presents acoustic feedback to its
      users using whistles. Future work can explore \pap ways to construct
      speakers, similar to those presented in \cite{Ishiguro:2014}, and provide
      richer acoustic output. Continuing, while with \al and \mp I explored
      different ways to provide visual output to users, future work can explore
      more intricate ways to visualize output using LEDs.
      
      Last, an interesting avenue to explore is the construction of tangible
      devices that not only respond to user's interactions, but to changes in
      the environment they sit in. The inclusion of ``smart'' materials can aid
      in the creation of such devices. The use of materials which properties
      change depending on their environment can enable the easy construction of
      ``tangible sensors''. These sensors can respond to environmental
      temperature, humidity, or sound, and act accordingly (e.g., a smart
      sunglass case that opens when it is bright out).

    \newpage
    \subsection{Support for other fabrication equipment}
      In this thesis I have presented four \pap ways to construct tangible,
      interactive devices using 3D printers. This does not mean, however, that
      \pap devices can only be fabricated using 3D printers. The main benefit of
      using 3D printers instead of other fabrication devices is their capability
      for constructing three-dimensional shapes without the need user
      intervention; whereas using other fabrication devices require manual
      assembly post-fabrication. A possible research avenue to construct
      tangible devices using laser cutters and CNC routers is to not only use
      these devices for the fabrication of the object, but for the assembly
      tasks as well.

      Despite early explorations in this concept have yielded promising
      results~\cite{Katakura:2019, Nisser:2021}, they remain far from the \papf
      ideal: \cite{Katakura:2019} is not able to construct tangible devices, but
      more objects that can be mechanically actuated, and \cite{Nisser:2021}
      requires intricate calibration of the laser cutter mid-print to solder
      conductive tracings, and combine previously cut parts. Future endeavors
      expanding this concept can explore the use of other materials, e.g.,
      conductive filaments, to simplify the connection between electronics.