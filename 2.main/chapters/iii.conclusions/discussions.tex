\chapter{Discussions \& Implications} \label{ch:discussion}
  This thesis has presented a set of techniques that enable the construction of
  interactive, tangible devices without requiring any domain experience (e.g.,
  assembly of printed parts and circuits, or calibration of machine learning
  models). To conclude, I discuss how each of the proposed techniques moves
  closer to the \papf ideal, and highlight some paths for future work.

  \section{Discussion of Projects}
    The main takeaway from this thesis is this: constructing tangible,
    interactive devices is hard---significantly harder than their on-screen
    counterparts---because to do so it is necessary to be well-versed in
    software, hardware, and digital fabrication. Ideally, digital fabrication
    equipment should abstract most, if not all, of these complications and
    enable designers to fabricate tangible, interactive devices that are
    immediately usable after construction. This ideal \red{scenario} is what I
    call ``\papf'': a fabrication paradigm where tangible, interactive devices
    are printed instead of assembled.

    In this thesis, I have presented four \pap techniques---\al, \bh, \al,
    \mp---each aimed to address specific barriers in the \red{interaction
    stack}: sensing, processing, and output. While this thesis has been aimed
    for the understanding of the general scientific community, the design and
    schematics of each of these projects have been open-sourced upon the
    publication of their respective manuscripts, enabling not only researchers
    but non-expert designers to experiment with the proposed techniques.

    I envision that, mirroring the process of creating graphical user
    interfaces, a \papf toolkit will abstract the interactive mechanisms from
    their respective techniques. For example, designers can specify ``a touch
    location here'', ``a \bh location there'', ``a linear actuator \red{here}'',
    and the system would create the desired tangible device, with the specified
    interaction modes. Continuing, once the tangible device has been designed
    successfully, this toolkit will then aid with the software side of things;
    providing pre-trained machine learning models to enable more experienced
    designers, or provide a programming-by-example environment where designers
    can link specific locations in the model, to actions.

    In addition to contributing to the digital fabrication \red{field}, the work
    I present in this thesis builds on the larger trend in computer science,
    specifically Human-Computer Interaction, where computing devices are imbued
    with domain expertise to support users. Although previous explorations of
    this concept in a digital fabrication setting have been aimed at preserving
    user creativity and agency~\cite{Zoran:2013}, the work discussed above aims
    to abstract the complexity of creating tangible devices into the fabrication
    process using 3D printers. By removing the complexities of constructing
    tangible, interactive devices and instead abstracting them into the
    fabrication pipelines, we can enable non-expert designers and hobbyists to
    create on-demand tangible devices without requiring extensive domain
    expertise, making the fabrication revolution ideal closer to reality.

    A potential \red{breakthrough} that might change the performance of the
    projects discussed above is the advent of new fabrication machines, or the
    increased accessibility of higher-end, current ones. While some of the
    barriers that previous endeavors face may be overcome by improvements in
    digital fabrication equipment (e.g., complex support material
    removal~\cite{Laput:2015}, or manual fabrication procedures~\cite{He:2017}),
    others efforts present intrinsic barriers in the way they enable the
    construction of tangible, interactive devices. For example, in
    Sauron~\cite{Savage:2013}, Savage \etal use video cameras placed in the
    interior of the object to track user's interactions.  Although fabrication
    methods will improve over time, to create Sauron-enabled objects, designers
    will be required to embed cameras, mirrors, and reflectors inside their
    fabricated objects, with all the complications this process entails. Another
    effort that presents similar barriers is
    \texttt{./trilaterate}~\cite{Schmitz:2019}, by Schmitz \etal While their
    fabricated objects are printed as a single structure using multi-material
    printers, the chosen way to identify user's interactions, trilateration,
    will require post-print calibration by user, and by object. This stands in
    contrast to the techniques included in this thesis, where the advent of new
    fabrication technologies will only improve their performance. For example,
    higher resolution printing can enable \at to embed more touch-sensitive
    locations throughout the model, breakthroughs in flexible materials can
    allow \mp to construct a wider range of shapes without the use of internal
    support, and the use of acoustic metamaterials like the ones presented
    in~\cite{Haberman:2016} could allow \bh to render a wider range of
    frequencies for its interactive locations.

    Last, in this thesis I have presented a series of projects that enable \papf
    using air-powered objects, via custom internal structures. The use of air as
    a driving force for these project was motivated by \red{two} main factors.
    First, \red{something}. Second, \red{something}. This does not mean that
    only air-powered approaches can be \pap. As mentioned previously, any
    technique that can allow the construction of tangible, interactive devices
    without the need of complex post-print activities, or prohibitive
    fabrication pipelines is a \papf technique. For example, recent work from
    Schmitz and colleagues present a very interesting way to tackle assembly:
    using magnets and conductive filament~\cite{Schmitz:2021}. The objects
    fabricated using this technique are constructed in two parts: an ``Oh Snap!
    board'' which houses a microcontroler with the logic, and the printed object
    itself.

  \section{Directions for Future Work}
    Throughout this thesis, I have laid out possible directions future
    researchers can improve each project. Moving from per-project improvements,
    this section aims to discuss how \papf can be improved by future
    explorations.

    \subsection{Support for multi-material printing}
      Throughout the projects that make up this thesis, I have made a conscious
      decision to construct tangible, interactive devices using single-material
      fabrication pipelines in order to increase the potential audience of each
      technique. Future efforts researching new \pap techniques for constructing
      tangible devices can explore the use of multi-material fabrication
      pipelines. Significant work has already been carried out by
      Schmitz~\cite{Schmitz:2019a} using a combination of conductive and
      nonconductive materials to construct tangible devices. While these efforts
      are not truly \pap, as they require post-print activities including
      carefully pouring liquids into the object or machine learning
      calibrations, they highlight the promising possibilities of using
      multi-material fabrication pipelines to construct interactive objects.

      In addition to using conductive and nonconductive material to construct
      tangible devices, future work can explore the use of metamaterials to
      create new composite materials, tailored for specific functions.  The use
      of multi-material fabrication pipelines has the potential for enriching
      the vocabulary of interaction modalities, while still remaining \pap.  For
      example, the use of \red{compressible} metamaterials can enable a variety
      of haptic feedback when interacting with the printed device, while
      \red{some other example here}.

    \subsection{Support for other interaction modalities}
      The work discussed above focuses on demonstrating the possibilities of
      constructing tangible, interactive devices with very little designer
      intervention post-print, rather than exploring a variety of interaction
      modalities. An interesting direction for future work is to investigate how
      the \papf principles can be materialized with various interaction
      modalities like deformation, and environmental interactions, as well as
      expand the interaction modalities already explored in this thesis.

      All the sensing techniques presented in this thesis are tailored for
      interaction on hard tangible devices. An interesting avenue for future
      work to explore is the implementation of \pap techniques that enable
      reliable deformation sensing in soft, and flexible objects. Regardless of
      the technique chosen to implement this interaction modality (e.g.,
      pneumatic, acoustic, or electric sensing), I believe the main challenge in
      implementing this interaction modality in a \pap way will be the removal
      of per-object calibration of machine learning models. Because of each
      object's varying geometry, interacting with different object should yield
      different results. Inspired by \at success in reusing machine learning
      models, future work can explore the use of custom, stable interior
      structures for deformation (e.g., press, squeeze) sensing.

      Continuing, I have presented two techniques for the construction of
      efforts that are able to respond to touch interactions: \at and \al. While
      successful, these efforts can only reliably sense single touch locations.
      I have laid out guidelines on how to pneumatically sense more than one
      location in Chapter \ref{ch:airtouch}, but this approach limits the number
      of interaction locations if a multi-touch setup is desired. Future
      explorations can investigate other \pap-friendly techniques for
      constructing tangible devices capable of sensing touch interaction in
      various locations.

      Conversely, with \al we explored ways to physicalize the output of logical
      operations. These outputs can also be expanded by future work. For
      example, \al presents a vibrotactile motor to provide haptic feedback to
      its users; future work can explore other haptic interaction modalities as
      output, like temperature. Similarly, \al presents acoustic feedback to its
      users using whistles. Future work can explore \pap ways to construct
      speakers, similar to those presented in \cite{Ishiguro:2014}, and provide
      richer acoustic output. Continuing, while with \al and \mp I explored
      different ways to provide visual output to users, future work can explore
      more intricate ways to visualize output using LEDs or \red{something else.}

      Last, an interesting avenue to explore is the construction of tangible
      devices that not only respond to user's interactions, but to changes in
      the environment they sit in. The inclusion of ``smart'' materials can aid
      in the creation of such devices. The use of materials which properties
      change depending on their environment can enable the easy construction of
      ``tangible sensors''. These sensors can respond to environmental
      temperature, humidity, or sound, and act accordingly (e.g., a smart
      sunglass case that opens when it is bright out).

    \subsection{Support for other fabrication equipment}
      In this thesis I have presented four \pap ways to construct tangible,
      interactive devices using 3D printers. This does not mean, however, that
      \pap devices can only be fabricated using 3D printers. The main benefit of
      using 3D printers instead of other fabrication devices is their capability
      for constructing three-dimensional shapes without the need user
      intervention; whereas using other fabrication devices require manual
      assembly post-fabrication. A possible research avenue to construct
      tangible devices using laser cutters and CNC routers is to not only use
      these devices for the fabrication of the object, but for the assembly
      tasks as well. 

      Early explorations of this concept have resulted in promising
      results~\cite{Katakura:2019, Nisser:2021}, however, they are still far
      from being \pap. For example, \cite{Katakura:2019} only enables the
      construction of mechanically interactive objects: objects that can be
      moved but don't sense how they're being used. And \cite{Nisser:2021} still
      requires the designer to be versed in the selection of electronics, and
      how to connect them. Future work expanding this concept should do
      \red{something else.}