\chapter*{Preface}\label{ch:preface}
  This thesis represents the culmination of four and a half years of independent
  research carried out in two universities. The main purpose of this thesis is
  to demonstrate that I was able to carry out international-level research
  during this period.

  \section*{Structure of Thesis}
    The main contribution of this thesis is the concept of \papf, which I
    present the three parts that make up this document. Part I aims to prepare
    the reader by introducing the concept of \papf, and providing background on
    the current position of the state-of-the-art for the fabrication of
    tangible, interactive devices. In Part II I present four novel techniques
    for fabricating tangible, interactive devices using consumer-grade 3D
    printers. Last, Part III offers high-level reflections about the projects
    that comprise this document, and outlines future steps for this avenue of
    research.

  \section*{Paper Selection}
    The core content of this thesis comes from four papers I have contributed
    to, of which three are published~\cite{Tejada:2018, Tejada:2020, Kim:2021a},
    and one is in manuscript~\cite{Tejada:}. Each of these papers introduces a
    novel technique for construction of tangible interactive devices requiring
    minimal post-fabrication activities, while focusing on a specific aspect of
    interactivity (sensing interactions, processing inputs, and output display).
    These papers all share the \papf ideal that tangible devices should be
    fabricated instead of assembled. Below, find abstracts for each of the
    papers that comprise this document.

  \subsection*{Paper Abstracts}
    \subsection*{\at: 3D-printed Touch-sensitive Objects Using Pneumatic
      Sensing}
      3D printing technology can be used to rapidly prototype the look and
      feel of 3D objects. However, the objects produced are passive. There
      has been increasing interest in making these objects interactive, yet
      they often require assembling components or complex calibration. In
      this paper, we contribute \at, a technique that enables designers
      to fabricate touch-sensitive objects with minimal assembly and
      calibration using pneumatic sensing. \at-enabled objects are 3D
      printed as a single structure using a consumer-level 3D printer.
      \at uses pre-trained machine learning models to identify
      interactions with fabricated objects, meaning that there is no
      calibration required once the object has completed printing. We
      evaluate our technique using fabricated objects with various geometries
      and touch sensitive locations, obtaining accuracies of at least 90\%
      with 12 interactive locations.

    \subsection*{\bh: Blowing-activated Tags for Interactive 3D-printed
      Models}
      Interactive 3D models have the potential to enhance accessibility and
      education, but can be complex and time-consuming to produce. We present
      \bh, a technique for embedding blowing-activated tags into
      3D-printed models to add interactivity. Requiring no special printing
      techniques, components, or assembly and working on consumer-level 3D
      printers, \bh adds acoustically resonant cavities to the interior
      of a model with unobtrusive openings at the surface of the object. A
      gentle blow into a hole produces a unique sound that identifies the
      hole, allowing a computer to provide associated content. We describe
      the theory behind \bh, characterize the performance of different
      cavity parameters, and describe our implementation, including
      easy-to-use software to automatically embed blowholes into preexisting
      models. We illustrate \bh's potential with multiple working
      examples.

    \subsection*{\al: A Toolkit for 3D-printing Stand-Alone, Interactive Objects}
      The promise of on-demand fabrication of custom, interactive devices is
      closer to reality thanks to recent developments in 3D-printing of
      interactive devices. While recent work has presented novel ways to
      3D-print artifacts such as speakers, electromagnetic actuators, and
      hydraulic robots; these efforts are non-trivial to instantiate,
      requiring assembly of circuits or mechanical parts. The present work
      introduces \al: a toolkit for the creation of stand-alone,
      interactive objects using pneumatic widgets. Objects constructed using
      \al, require no electronic circuits, and little to no assembly of
      physical components. \al is comprised of a set of 12 pneumatic
      widgets, and a design environment, which designers can use to embed
      input, logic processing, and output capabilities to existing 3D models.
      We present an evaluation of the performance of our widgets, and a four
      applications that illustrate \al's potential.

    \subsection*{\mp: A Toolkit for Prototyping Shape-Changing
      Interfaces}
      Toolkits for shape-changing interfaces (SCIs) enable designers and
      researchers to easily explore the broad design space of SCIs. However,
      despite their utility, existing approaches are often limited in the
      number of shape-change features they can express. This paper introduces
      \mp, a toolkit for creating SCIs that covers seven out of the
      eleven shape-change features. \mp is comprised of (1) a set
      of six standardized widgets that express the shape-change features with
      user-definable parameters; (2) software for 3D modeling the widgets to
      create 3D-printable pneumatic SCIs; and (3) a hardware platform for
      controlling the widgets. To evaluate \mp we carried out ten
      open-ended interviews with novice and expert designers who were asked
      to design a SCI using our software. Participants highlighted the ease
      of use and expressivity of the \mp.