\chapter*{Preface}\label{ch:preface}
Lorem ipsum dolor sit amet, consectetur adipiscing elit. Fusce iaculis
tristique est, ac lobortis erat congue quis. Cras scelerisque leo diam, non
porta massa interdum et. Donec dolor metus, porttitor quis posuere sed,
sagittis nec risus. Sed sit amet scelerisque justo. In tempus lectus sed
pulvinar fermentum. Morbi lobortis ut dui sit amet porta. In nulla tortor,
ultrices nec ligula vitae, consequat iaculis leo. Nam a ante ut elit dapibus
maximus. Mauris porta odio dictum ligula pretium commodo. Nullam interdum
neque lorem, sed molestie leo auctor hendrerit. Nam finibus nisi eu mauris
tincidunt, ut vulputate enim mattis. Proin pulvinar vehicula mauris.

Mauris iaculis, risus sed tincidunt lacinia, odio lectus laoreet enim, at
ultrices libero metus eget mi. Proin cursus congue finibus. Donec facilisis
at eros a eleifend. Fusce nec tincidunt ligula. Duis consectetur urna erat,
nec pulvinar massa eleifend at. Donec in nulla justo. Aliquam luctus luctus
mauris non ornare.

  \section*{Structure of Thesis}
    Lorem ipsum dolor sit amet, consectetur adipiscing elit. Mauris in quam nec
    lacus consectetur malesuada a eu lectus. Fusce a imperdiet turpis, ac
    consequat sem. Class aptent taciti sociosqu ad litora torquent per conubia
    nostra, per inceptos himenaeos. In fermentum vulputate augue vel auctor. Ut
    ante diam, finibus quis lorem in, pretium elementum nunc. Sed sollicitudin
    libero felis, sit amet venenatis nibh mollis a. Sed et est vitae leo
    fermentum ornare. Vestibulum odio nulla, imperdiet ac purus ut, malesuada
    vulputate erat. Vivamus id tempor nisi, porta semper est. Cras quis elit
    euismod, lacinia eros id, porttitor magna. Ut volutpat quis ipsum et
    rhoncus.

    Duis eu pulvinar lacus. Cras commodo feugiat urna ac auctor. Aliquam nec
    blandit nisl. In ut faucibus orci. Nam viverra dolor eget egestas lobortis.
    Proin ac risus vel purus sodales convallis a vitae dolor. In pretium varius
    diam, eu consequat quam scelerisque non. Phasellus pulvinar, mauris id
    bibendum malesuada, purus nibh vulputate velit, vel ultrices metus justo
    sit amet urna. Phasellus quis enim nec odio dignissim suscipit.

  \section*{Paper Selection}
    Lorem ipsum dolor sit amet, consectetur adipiscing elit. Nam tempus lacus eu
    rutrum vehicula. Duis convallis justo ac blandit egestas. Sed quis
    scelerisque nibh. Sed lacus odio, tristique vitae egestas nec, aliquet auctor
    nisi. Mauris porta at nibh eget maximus. Sed sed erat non nulla iaculis
    dapibus. In ultrices blandit tellus, ut ultricies tellus molestie id. Nam id
    commodo est, quis posuere elit. Integer nisi odio, tincidunt vel magna
    feugiat, tristique luctus lorem.

  \section*{Paper Abstracts}
    \subsection*{Blowhole: Blowing-activated Tags for Interactive 3D-printed
      Models}
      Interactive 3D models have the potential to enhance accessibility and
      education, but can be complex and time-consuming to produce. We present
      Blowhole, a technique for embedding blowing-activated tags into
      3D-printed models to add interactivity. Requiring no special printing
      techniques, components, or assembly and working on consumer-level 3D
      printers, Blowhole adds acoustically resonant cavities to the interior
      of a model with unobtrusive openings at the surface of the object. A
      gentle blow into a hole produces a unique sound that identifies the
      hole, allowing a computer to provide associated content. We describe
      the theory behind Blowhole, characterize the performance of different
      cavity parameters, and describe our implementation, including
      easy-to-use software to automatically embed blowholes into preexisting
      models. We illustrate Blowhole's potential with multiple working
      examples.

    \subsection*{Airtouch: 3D-printed Touch-sensitive Objects Using Pneumatic
      Sensing}
      3D printing technology can be used to rapidly prototype the look and
      feel of 3D objects. However, the objects produced are passive. There
      has been increasing interest in making these objects interactive, yet
      they often require assembling components or complex calibration. In
      this paper, we contribute AirTouch, a technique that enables designers
      to fabricate touch-sensitive objects with minimal assembly and
      calibration using pneumatic sensing. AirTouch-enabled objects are 3D
      printed as a single structure using a consumer-level 3D printer.
      AirTouch uses pre-trained machine learning models to identify
      interactions with fabricated objects, meaning that there is no
      calibration required once the object has completed printing. We
      evaluate our technique using fabricated objects with various geometries
      and touch sensitive locations, obtaining accuracies of at least 90\%
      with 12 interactive locations.

    \subsection*{MorpheesPlug: A Toolkit for Prototyping Shape-Changing
      Interfaces}
      Toolkits for shape-changing interfaces (SCIs) enable designers and
      researchers to easily explore the broad design space of SCIs. However,
      despite their utility, existing approaches are often limited in the
      number of shape-change features they can express. This paper introduces
      MorpheesPlugs, a toolkit for creating SCIs that covers seven out of the
      eleven shape-change features. MorpheesPlugs is comprised of (1) a set
      of six standardized widgets that express the shape-change features with
      user-definable parameters; (2) software for 3D modeling the widgets to
      create 3D-printable pneumatic SCIs; and (3) a hardware platform for
      controlling the widgets. To evaluate MorpheesPlugs we carried out ten
      open-ended interviews with novice and expert designers who were asked
      to design a SCI using our software. Participants highlighted the ease
      of use and expressivity of the MorpheesPlug.

    \subsection*{3D-printed Logic}
      The promise of on-demand fabrication of custom, interactive devices is
      closer to reality thanks to recent developments in 3D-printing of
      interactive devices. While recent work has presented novel ways to
      3D-print artifacts such as speakers, electromagnetic actuators, and
      hydraulic robots; these efforts are non-trivial to instantiate,
      requiring assembly of circuits or mechanical parts. The present work
      introduces AirLogic: a toolkit for the creation of stand-alone,
      interactive objects using pneumatic widgets. Objects constructed using
      AirLogic, require no electronic circuits, and little to no assembly of
      physical components. AirLogic is comprised of a set of 12 pneumatic
      widgets, and a design environment, which designers can use to embed
      input, logic processing, and output capabilities to existing 3D models.
      We present an evaluation of the performance of our widgets, and a four
      applications that illustrate AirLogic's potential.