% Abstract
\pdfbookmark[1]{Abstract}{Abstract}
\begingroup
	\let\clearpage\relax
	\let\cleardoublepage\relax
	\let\cleardoublepage\relax

	\chapter*{Abstract}
		In recent years, it has become increasingly accessible to create interactive
		applications on screen-based devices. Contrary to this ease, and despite
		their numerous benefits, creating tangible interactive devices is a tasks
		reserved for experts, requiring extensive knowledge on electronics, and
		manual assemblies. While digital fabrication equipment holds promise to
		alleviate this situation, the majority of research exploring this avenue
		still present significant barriers for non-experts, and other-domain experts
		to construct tangible devices, often requiring assembly of electronic
		circuits and printed parts, prohibitive fabrication pipelines, or intricate
		calibration of machine learning models.

		This thesis introduces \emph{\papf}: a digital fabrication paradigm where
		tangible interactive devices are printed, rather than assembled. By
		embedding interior structures inside three-dimensional models that leverage
		distinct properties of fluid behavior, this thesis presents a variety of
		techniques to construct tangible devices that can sense, process, and
		respond to user's interactions without requiring assembly of parts,
		circuits, or calibration of machine learning models.

		Chapter \ref{ch:background} provides an overview of the fabrication of
		tangible devices literature through the lens of \papf. This chapter
		highlights the post-print activities required to enable each of the efforts
		in the literature, and reflects on the status of the field.

		Chapters \ref{ch:airtouch} and \ref{ch:blowhole} introduce two novel
		techniques for constructing tangible devices that can sense user's
		interactions. \at uses basic principles of fluid behavior to enable the
		construction of touch-sensing devices, capable of detecting interactions in
		up to 12 locations, with an accuracy of up to 98\%. \bh builds on this
		concept by employing principles of acoustic resonance to construct tangible
		devices that can detect where are gently blown on. \bh-enabled devices can
		enable up to seven interactive locations, with an accuracy of up to 98\%.

		Conversely, in Chapter \ref{ch:airlogic} I introduce a technique to
		encapsulate logic computation into 3D-printed objects. Inspired by concepts
		from the Cold War, I embed structures capable of representing basic logic
		operations using interacting jets of air into three-dimensional models. \al
		takes the form of a toolkit, enabling non-expert designers to add a variety
		of input mechanisms, logic operations, and output displays.

		Continuing, Chapter \ref{ch:morphees} describes a toolkit for fabricating
		objects capable of changing their physical shape using pneumatic actuation.
		\mp introduces a design environment, a set of pneumatically actuated
		widgets, and a control module that, in tandem, enable non-experts to
		construct devices capable of changing their physical shape in order to
		provide output.

		Last, I conclude with reflections on the status of \papf, and possible
		directions for future work.

		\newpage

		\begin{otherlanguage}{danish}
			\pdfbookmark[1]{Dansk Resum\'e}{Dansk Resum\'e}
			\chapter*{Dansk Resum\'e}
			Something in Danish\dots
		\end{otherlanguage}

\endgroup

	\vfill
