% Abstract
\pdfbookmark[1]{Abstract}{Abstract}
\begingroup
	\let\clearpage\relax
	\let\cleardoublepage\relax
	\let\cleardoublepage\relax

	\chapter*{Abstract}
		In recent years, it has become increasingly \red{easy} to create interactive
		applications on screen-based devices. Contrary to this ease, and despite
		their numerous benefits, creating tangible interactive devices is a tasks
		reserved for experts, requiring extensive knowledge on electronics. \red{A
		promising solution are digital fabrication technologies, but significant
		barriers still stand between non-experts, and other-domain experts when
		constructing tangible devices.} 

		This thesis introduces \emph{\papf}: a fabrication paradigm where tangible
		interactive devices are printed, rather than assembled. By embedding
		interior structures inside three-dimensional models that leverage distinct
		properties of fluid behavior, this thesis presents a variety of techniques
		to construct tangible devices that can sense, process, and respond to user's
		interactions without requiring assembly of parts, circuits, or calibration
		of machine learning models.

		% Maybe expand each paper
		Chapters \ref{ch:airtouch} and \ref{ch:blowhole} introduce two novel
		techniques for constructing tangible devices that can sense user's
		interactions. \at uses basic principles of fluid behavior to enable the
		construction of touch-sensing devices, whereas \bh employs principles of
		acoustic resonance to identify where objects are being gently blown on.

		Conversely, in Chapter \ref{ch:airlogic} I introduce a technique to
		encapsulate logic computation into 3D-printed objects. Inspired by concepts
		from the Cold War, I embed structures capable of representing basic logic
		operations using interacting jets of air into three-dimensional models.

		Continuing, Chapter \ref{ch:morphees} describes a toolkit for fabricating
		objects capable of changing their physical shape using pneumatic actuation.
		\mp introduces a set of widgets,

		Last, I conclude with reflections on the status of \papf, and possible
		directions for future work.

		\newpage

		\begin{otherlanguage}{danish}
			\pdfbookmark[1]{Dansk Resum\'e}{Dansk Resum\'e}
			\chapter*{Dansk Resum\'e}
			Something in Danish\dots
		\end{otherlanguage}

\endgroup

	\vfill
