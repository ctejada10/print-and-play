% Abstract
\pdfbookmark[1]{Abstract}{Abstract}
\begingroup
	\let\clearpage\relax
	\let\cleardoublepage\relax
	\let\cleardoublepage\relax

	\chapter*{Abstract}
		In recent years, it has become increasingly accessible to create interactive
		applications on screen-based devices. Contrary to this ease, and despite
		their numerous benefits, creating tangible interactive devices is a task
		reserved for experts, requiring extensive knowledge on electronics, and
		manual assemblies. While digital fabrication equipment holds promise to
		alleviate this situation, the majority of research exploring this avenue
		still present significant barriers for non-experts, and other-domain experts
		to construct tangible devices, often requiring assembly of electronic
		circuits and printed parts, prohibitive fabrication pipelines, or intricate
		calibration of machine learning models.

		This thesis introduces \emph{\papf}: a digital fabrication paradigm where
		tangible interactive devices are printed, rather than assembled. By
		embedding interior structures inside three-dimensional models that leverage
		distinct properties of fluid behavior, this thesis presents a variety of
		techniques to construct tangible devices that can sense, process, and
		respond to user's interactions without requiring assembly of parts,
		circuits, or calibration of machine learning models.

		Chapter \ref{ch:background} provides an overview of the fabrication of
		tangible devices literature through the lens of \papf. This chapter
		highlights the post-print activities required to enable each of the efforts
		in the literature, and reflects on the status of the field.

		Chapters \ref{ch:airtouch} and \ref{ch:blowhole} introduce two novel
		techniques for constructing tangible devices that can sense user's
		interactions. \at uses basic principles of fluid behavior to enable the
		construction of touch-sensing devices, capable of detecting interactions in
		up to 12 locations, with an accuracy of up to 98\%. \bh builds on this
		concept by employing principles of acoustic resonance to construct tangible
		devices that can detect where they are gently blown on. \bh-enabled devices
		can enable up to seven interactive locations, with an accuracy of up to
		98\%.

		Conversely, in Chapter \ref{ch:airlogic} I introduce a technique to
		encapsulate logic computation into 3D-printed objects. Inspired by concepts
		from the Cold War era, I embed structures capable of representing basic
		logic operations using interacting jets of air into three-dimensional
		models. \al takes the form of a toolkit, enabling non-expert designers to
		add a variety of input, logic processing, and output mechanisms to
		three-dimensional models.

		Continuing, Chapter \ref{ch:morphees} describes a toolkit for fabricating
		objects capable of changing their physical shape using pneumatic actuation.
		\mp introduces a design environment, a set of pneumatically actuated
		widgets, and a control module that, in tandem, enable non-experts to
		construct devices capable of changing their physical shape in order to
		provide output.

		Last, I conclude with reflections on the status of \papf, and possible
		directions for future work.

		\newpage

		\begin{otherlanguage}{danish}
			\pdfbookmark[1]{Dansk Resum\'e}{Dansk Resum\'e}
			\chapter*{Dansk Resum\'e}
				I de senere år er det blevet mere og mere tilgængeligt at skabe
				interaktive applikationer på skærmbaserede apparater. I modsætning til
				denne tilgængelighed, og til trods for deres mange fordele, er skabelsen
				af såkaldte ”tangible devices” en opgave som kun kan udføres af eksperter,
				da det kræver omfattende viden om elektronik og manual samling. Selvom
				digital fabrikation er lovende for at afhjælpe denne situation, indeholder
				størstedelen af forskningen i dette felt stadigvæk store barrierer for
				ikke-eksperter og for eksperter fra andre domæner, når det kommer til
				skabelsen af tangible devices, da det ofte kræver samling af elektroniske
				kredsløb og printede dele, uoverkommelige fabrikationsprocesser, og
				indviklet kalibrering af maskinlæringsmodeller.
				
				Denne afhandling introducerer \emph{\papf}: et digitalt
				fabrikationsparadigme hvor interaktive tangible devices bliver printet i
				stedet for samlet. Ved at indsætte indre strukturer i tredimensionelle
				modeller som udnytter egenskaber fra væskedynamik præsenterer afhandlingen
				en vifte af teknikker til at konstruere tangible devices som kan føle,
				beregne, og besvare brugeres interaktioner uden brug af dele, kredsløb,
				eller kalibrering af maskinlæringsmodeller.
				
				Kapitlet \ref{ch:background} giver et overblik over fabrikationen af
				litteraturen om tangible devices gennem et perspektiv som tager
				udgangspunkt i \emph{\papf}. Dette kapitel fremhæver de aktiviteter, man
				skal udføre efter at have printet, som er påkrævet for at opnå, og
				reflekterer på feltets status.
				
				Kapitlerne \ref{ch:airtouch} og \ref{ch:blowhole} introducerer to nye
				teknikker til at konstruere tangible devices som kan føle brugerens
				interaktioner. \at udnytter grundlæggende væskedynamiksprincipper for at
				skabe apparater som kan måle berøring ved op til 12 lokationer med en
				nøjagtighed på op til 98\%. \bh bygger på dette koncept ved at bruge
				principper fra akustisk resonans til at konstruere tangible devices som
				kan måle når brugeren puster mildt på dem. Apparater som anvender \bh
				muliggør op til syv interaktive lokationer, med en nøjagtighed på op til
				98\%.
				
				Omvendt introducerer jeg i kapitel \ref{ch:airlogic}, en teknik som
				indkapsler logikberegning I 3D-printede objekter. Inspireret af koncepter
				fra koldkrigstiden indsætter jeg strukturer som kan repræsentere
				grundlæggende logikoperationer ved hjælp af luftstråler i tredimensionelle
				modeller. \al tager form af et toolkit, der tillader designere som ikke er
				eksperter at tilføje en vifte af inputmekanismer, logikoperationer, og
				outputmekanismer til tredimensionelle modeller.
				
				Videre beskriver kapitel \ref{ch:morphees} et toolkit for fabrikerede
				objekter som kan ændre sin fysiske form gennem pneumatisk aktivering. \mp
				introducerer et designmiljø, et sæt af pneumatisk aktiverede widgets, og
				et kontrolmodul som samlet tillader ikke-eksperter at konstruere apparater
				som kan ændre sin fysiske form for at give output.
				
				Slutteligt konkluderer jeg med refleksioner om status på \papf, og mulige
				retninger for det fremtidige arbejde.
		\end{otherlanguage}

\endgroup

	\vfill
